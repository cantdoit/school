% !TEX TS-program = XeLaTeX
%% compile with xelatex sudo easy_install Pygments
\documentclass[11pt,a4paper]{article}

\usepackage[UTF8]{ctex}
\usepackage{homework}

\title{作业 1}
\duedate{2024 年 3 月 28 日}
% TODO your name and ID
\studentname{Johnson 许霆康}
\studentid{2019080126}

\begin{document}

\maketitle

\textit{在开始完成作业前,请仔细阅读以下说明:}
\begin{itemize}
    \item 我们提供作业的\LaTeX 源码,你可以在其中直接填充你的答案并编译PDF(请使用xelatex)。
    当然,你也可以使用别的方式完成作业(例如撰写纸质作业后扫描到PDF文件之中)。
    但是请注意,最终的提交一定只是PDF文件。提交时请务必\emph{再次核对},防止提交错误。
    \item 在你的作业中,请务必填写你的\emph{姓名}和\emph{学号},并检查是否有题目遗漏。请重点注意每次作业的\emph{截止时间}。
    截止时间之后你仍可以联系助教补交作业,但是我们会按照如下公式进行分数的折扣:
    \begin{align*}
        \text{作业分数} = \text{满分}\times\left(1 - 10\%\times\min\left(\lceil\text{迟交周数}\rceil, 10\right)\right)\times\text{正确率}.
    \end{align*}
    \item 本次作业为独立作业,\emph{禁止}抄袭等一切不诚信行为。作业中,如果涉及参考资料,请\emph{引用}注明。
\end{itemize}

%% problem begins

\problem{判断题}

给定下列陈述,请判断其是否正确。如果错误,请给出反例或解释原因。

\subproblem 给定任意的命题逻辑公式,它是否为永假式一定是可判定的。

\begin{solution}
    % DONE
    True.
\end{solution}

\subproblem 给定命题逻辑公式$F$和$G$,如果$F$是有效的且$G$不是有效的,
则$F\rightarrow G$一定不可满足。

\begin{solution}
    % DONE
    False.
    $F$ is always true since it is valid and $G$ contains some outcome which can be true since it is not valid. Therefore $F \rightarrow G$ is satisfiable.
\end{solution}

\subproblem 给定命题逻辑公式$F$和$G$,如果$F$是可满足的且$\neg G$是不可满足的,则$F\wedge G$一定可满足。

\begin{solution}
    % DONE
    True.
    $F$ is satisfiable and $G$ is valid. 
\end{solution}

\subproblem 任意给定一个一阶逻辑公式,一定可以在有限时间内判定其是否有效。

\begin{solution}
    % DONE
    True.
\end{solution}


% \newpage
\problem{解答题}
\subproblem 考虑下列公式(记作F):
$$(P \lor Q \to R) \to (\lnot R \lor \lnot P) \to (Q \to R)$$
请列出它的真值表,并判断它是否有效. 若有效,请使用$\mathcal{S}_{PL}$证明其有效性,即给出$\vdash F$的证明.

\begin{solution}
    % DONE damn so hard to type
\[
\begin{array}{|c|c|c|c|c|c|c|c|}
\hline
P & Q & R & (P \lor Q \rightarrow R) & (\neg R \lor \neg P) & (Q \rightarrow R) & (P \lor Q \rightarrow R) \rightarrow (\neg R \lor \neg P) \rightarrow (Q \rightarrow R) \\
\hline
\text{T} & \text{T} & \text{T} & \text{T} & \text{F} & \text{T} & \text{T} \\
\text{T} & \text{T} & \text{F} & \text{F} & \text{T} & \text{F} & \text{T} \\
\text{T} & \text{F} & \text{T} & \text{T} & \text{F} & \text{T} & \text{T} \\
\text{T} & \text{F} & \text{F} & \text{F} & \text{T} & \text{T} & \text{T} \\
\text{F} & \text{T} & \text{T} & \text{T} & \text{T} & \text{T} & \text{T} \\
\text{F} & \text{T} & \text{F} & \text{F} & \text{T} & \text{F} & \text{T} \\
\text{F} & \text{F} & \text{T} & \text{T} & \text{T} & \text{T} & \text{T} \\
\text{F} & \text{F} & \text{F} & \text{T} & \text{T} & \text{T} & \text{T} \\
\hline
\end{array}
\]
\[
\infer[\text{($\rightarrow R$)}]
{\vdash (P \lor Q \rightarrow R) \rightarrow (\neg R \lor \neg P) \rightarrow (Q \rightarrow R)}
{
    \infer[\text{($\rightarrow R$)}]
    {P \lor Q \rightarrow R \vdash (\neg R \lor \neg P) \rightarrow (Q \rightarrow R)}
    {
        \infer[\text{($\rightarrow R$)}]
        {P \lor Q \rightarrow R, \neg R \lor \neg P \vdash Q \rightarrow R}
        {
        	    \infer[\text{($\rightarrow L$)}]
	    {P \lor Q \rightarrow R, \neg R \lor \neg P ,Q \vdash R}
	    {
	        \infer[\text{($\neg R$)}]
	        {P \lor Q \rightarrow R, Q \vdash \neg R, R}	       
	            {
	            	\infer[\text{($Ax$)}]
	            	{P \lor Q \rightarrow R, Q, R \vdash R}
	            }
	        \quad
	        \infer[\text{($\rightarrow L$)}]
	        {P \lor Q \rightarrow R , \neg P, Q \vdash R}
	            {
	            	\infer[\text{($\rightarrow R$)}]
	                {\neg P, Q \vdash P \lor Q, R}
	                    {
	                        \infer[\text{($Ax$)}]
	                        {\neg P, P, Q \vdash Q, R}
	                    }
	                \quad
	                \infer[\text{($Ax$)}]
	                {\neg P, Q, R \vdash R}
	                {}
	            }
	    }
        }
    }
}
\]
\end{solution}

\subproblem 证明$\mathcal{S}_{PL}$的切规则是可靠的:

$$
\infer[\text{(切)}]
{\Gamma \vdash \Delta}
{\Gamma \vdash C, \Delta \qquad \Gamma,C \vdash \Delta}
$$

\begin{solution}
    % DOne
    Proof by Induction such that from height 1 to n where there is an instance where there exists an Axiom which shows it is satisfiable.
    
    Assuming we have the following:

$$
\infer[\text{(cut)}]
{\Gamma, A \vdash \Delta}
{\Gamma, A\vdash A, \Delta \qquad \Gamma, A \vdash \Delta}
$$
$${\big\downarrow}$$
$$
\infer[\text{(cut)}]
{\Gamma, A \vdash \Delta}
{
    \infer[\text{(Axiom)}]
    {\Gamma, A\vdash A, \Delta}
	\qquad
	\infer 
	{\Gamma, A \vdash \Delta}
	{\bigtriangledown}
	}
$$

Since from this get an Axiom it means that it would be satisfiable as long as we can reach an instance of {\Gamma, A \vdash \Delta}
    
    Assuming the previous as Part 1, now returning to the initial equation:
    
$$
\infer[\text{(cut)}]
{\Gamma \vdash \Delta}
{\Gamma \vdash B, \Delta \qquad \Gamma, B \vdash \Delta}
$$
$${\big\downarrow}$$
$$
\infer[\text{(cut)}]
{\Gamma \vdash \Delta}
{
    \infer[\text{}]
    {\Gamma, \vdash B, \Delta}
    {
        {\bigtriangledown}
    }
	\qquad
	\infer[\text{(cut)}]
	{\Gamma, B \vdash \Delta}
	{\text{Part 1}}
	}
$$
We can see that cut would give us Part 1 which means there exists an Axiom and therefore for any {\Gamma \vdash \Delta}, 
\text{there exists a cut which results in a solution therefore proving it is satisfiable.}
    
\end{solution}

\subproblem 考虑论域$\domain = \{\circ, \bullet\}$以及下面的解释函数
\begin{itemize}
  \item $\interpretation(f) = \{(\circ, \circ)\mapsto \bullet, (\circ, \bullet)\mapsto \circ, (\bullet, \circ)\mapsto \bullet, (\bullet, \bullet)\mapsto \circ\}$
  \item $\interpretation(g) = \{\circ \mapsto \bullet, \bullet \mapsto \circ\}$
  \item $\interpretation(p) = \{(\bullet, \circ), (\circ, \bullet)\}$
\end{itemize}
求公式$\forall x.p(f(g(x), x), x)$的取值。

\begin{solution}
    % TODOne
    \begin{itemize}
    \item $Let {\llbracket x \rrbracket}_{\mathcal{M},\rho} = \circ$
    \item ${\llbracket g(x) \rrbracket}_{\mathcal{M},\rho} = \interpretation(g)({\llbracket x \rrbracket}_{\mathcal{M},\rho}) = \interpretation(g)(\circ) = \bullet$
    \item ${\llbracket f(g(x), x) \rrbracket}_{\mathcal{M},\rho} = \interpretation(f)(\bullet, \circ)$ = \bullet$
    \item ${\llbracket p(f(g(x), x), x) \rrbracket}_{\mathcal{M},\rho} = \interpretation(p)(\bullet, \circ) = true$
    \item $Let {\llbracket x \rrbracket}_{\mathcal{M},\rho} = \bullet$
    \item ${\llbracket g(x) \rrbracket}_{\mathcal{M},\rho} = \interpretation(g)({\llbracket x \rrbracket}_{\mathcal{M},\rho}) = \interpretation(g)(\bullet) = \circ$
    \item ${\llbracket f(g(x), x) \rrbracket}_{\mathcal{M},\rho} = \interpretation(f)(\circ, \bullet)$ = \circ$
    \item ${\llbracket p(f(g(x), x), x) \rrbracket}_{\mathcal{M},\rho} = \interpretation(p)(\circ, \bullet) = true$
    \item For all of $x = \{\circ, \bullet\}$ is True therefore $\forall x.p(f(g(x), x), x)$ is True.
    \end{itemize}
\end{solution}

\subproblem 请使用$\mathcal{S}_{FOL}$(包含命题逻辑中的10条规则和4条量词消去规则)构建推导树证明下列两个相继式:

\begin{enumerate}
    \item $\exists x.(p(x)\rightarrow q(x)) \vdash \forall y.p(y) \rightarrow \exists z.q(z)$
    \item $\forall y.p(y)\rightarrow \exists z.q(z) \vdash \exists x.(p(x)\rightarrow q(x))$
\end{enumerate}

\begin{solution}
    % TODOne Iam going to krill myself
    
    $\exists x.(p(x)\rightarrow q(x)) \vdash \forall y.p(y) \rightarrow \exists z.q(z)$
    $$
    \infer[\text{($\rightarrow R$)}]
    {\exists x.(p(x)\rightarrow q(x)) \vdash \forall y.p(y) \rightarrow \exists z.q(z)}
    {
    	\infer[\text{($\neg R$)}]
   	 {\exists x.(p(x)\rightarrow q(x)), \forall y.p(y) \vdash \exists z.q(z)}
	 {
	 	\infer[\text{($\exists L$)}]
	 	{\exists x.(p(x)\rightarrow q(x)), \forall y.p(y), \neg \exists z.\neg q(z) \vdash \perp}
		{
			\infer[\text{($\neg L$)}]
			{\exists x.(p(x)\rightarrow q(x)), \forall y.p(y), \neg q(c) \vdash \perp}
			{
				\infer[\text{($\neg L$)}]
				{\exists x.(p(x)\rightarrow q(x)), \forall y.p(y), \vdash q(c)}
				{
					\infer[\text{($\forall R$)}]
					{\exists x.(p(x)\rightarrow q(x)) \vdash \neg \forall y.\neg p(y), q(c)}
					{
						\infer[\text{($\exists L$)}]
						{\exists x.(p(x)\rightarrow q(x)) \vdash \neg p(c), q(c)}
						{
							\infer[\text{($\rightarrow L$)}]
							{p(c) \rightarrow q(c)) \vdash \neg p(c),q(c)}
							{
							\infer[\text{($\neg L$)}]
							{\vdash p(c) \neg p(c), q(c)}
							{
								\infer[\text{(Ax)}]
								{p(c) \vdash p(c), q(c)}
							}
							\quad
							\infer[\text{(Ax)}]
							{q(c) \vdash q(c), \neg p(c)}
							}
						}
					}
				}
			}
		}
	 }
	 }
    $$
    
   $\forall y.p(y)\rightarrow \exists z.q(z) \vdash \exists x.(p(x)\rightarrow q(x))$
   $$
   \infer[\text{($\neg R$)}]
   {\forall y.p(y)\rightarrow \exists z.q(z) \vdash \exists x.(p(x)\rightarrow q(x))}
   {
   	\infer[\text{($\exists L$)}]
	{\forall y.p(y)\rightarrow \exists z.q(z)  \neg \exists x.\neg (p(x)\rightarrow q(x)) \vdash \perp}
	{
		\infer[\text{($\neg L$)}]
		{\forall y.p(y)\rightarrow \exists z.q(z), \neg (p(c)\rightarrow q(c)) \vdash \perp}
		{
			\infer[\text{($\rightarrow R$)}]
			{\forall y.p(y)\rightarrow \exists z.q(z) \vdash p(c)\rightarrow q(c)}
			{
				\infer[\text{($\rightarrow L$)}]
				{\forall y.p(y)\rightarrow \exists z.q(z), p(c) \vdash q(c)}
				{
					\infer[\text{($\forall R$)}]
					{p(c) \vdash \forall y.p(y), q(c)}
					{	
						\infer[\text{(Ax)}]
						{p(c) \vdash p(c), q(c)}
					}
				\quad
					\infer[\text{($\exists L$)}]
					{p(c), \exists z.q(z) \vdash q(c)}
					{
						\infer[\text{(Ax)}]
						{p(c), q(c) \vdash q(c)}
					}
				}
			}
		}
	}
   }
   
   $$

\end{solution}

\end{document}
